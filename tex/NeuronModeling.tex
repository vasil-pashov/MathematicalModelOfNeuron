\documentclass{article}

\usepackage[utf8]{inputenc}
\usepackage[english,bulgarian]{babel}

\title{Приложение на математиката за моделиране на реални процеси. Моделиране на неврон.}
\author{Тонка Желева \and Васил Пашов}

\begin{document}

\maketitle
\newpage

\tableofcontents
\newpage

1. Науката за невроните и приложение
2. Структура на неврона
  1. Обобщено устройство на неврона
  2. Синапси
3. Мембрана на неврона
4. Аксонът на Ходжкинс-Хъксли
5. Извеждане на математическия модел и метод за решаване
6. Програмна реализация на математическия модел и анимация

\section{Науката за невроните и приложение}

Мозъкът, намиращ се в черепната кухина, е част от централната нервна ссистема. Той е основният орган, който обработва всички съзнателни и несъзнателни стимули, чувства,  познания и памет. Също така е отговрен за контролирането на множество други органи. Основната градивна единица на мозъка е неврона. Човешкия мозък съдържа около   неврона. Невронните клетки приемат, обработват и изпращат нервния импулс.
Животните реагират на външни въздействия, използвайки невроните.  Например при заплаха от изгаряне, рецепторите за топлина на сензорен неврон осъществяват връзка със стимула и изпращат информация до интер-неврон в централната нервна система. От там мото-неврон изпраща отговорът до скелетните мускули, които карат тялото да се отдръпне. В основата на извършването на този процес стои невротрансмисията, която се извършва във всични неврони в човешкото тяло. Невроните пренасят тази информация чрез промени в 
електричесния потенциал на мембраната.

\section{Устройство на неврона}

аксон - най-издълженият израстък на неврона, чиято дължина може да надхвърли десетки хиляди пъти диаметъра на клетъчното тяло. Аксонът извежда нервните импулси от клетъчното тяло, пренасяйки информация до друга клетка. Нервните импулси са еднопосочни в аксонът, но невронът може да получи информация под формата на протеини които се придвижват от синапса до клетъчното ядро. Много неврони имат само един аксон, но той се разклонява в много направления и така прави възможна комуникацията с много клетки.
нервен импулс – 
\end{document} 
